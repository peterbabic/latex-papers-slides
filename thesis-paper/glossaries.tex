%\newacronym{$\upmu$}{mikro, $10^{-6}$}
\newacronym{io}{I/O}{Input/Output}
\newacronym{rom}{ROM}{Read-Only memory}
\newacronym{ram}{RAM}{Random-access memory}
\newacronym{gpio}{GPIO}{General-purpose \acrshort{io}}
\newacronym{hz}{Hz}{Hertz, the SI unit of frequency}
\newacronym{mhz}{MHz}{Mega-Hertz}
\newacronym{ghz}{GHz}{Giga-Hertz}
\newacronym{si}{SI}{Syst\`eme International}
\newacronym{soc}{SoC}{System-on-Chip}
\newacronym{wlan}{WLAN}{Wireless local area network}
\newacronym{ap}{AP}{Access Point}
\newacronym{ieee}{IEEE}{Institute of Electrical and Electronics Engineers}
\newacronym{uart}{UART}{Universal asynchronous receiver/transmitter}
\newacronym{led}{LED}{Light emitting diode}
\newacronym{v}{V}{volt, the SI unit of electric potential}
\newacronym{mips}{MIPS}{Microprocessor without Interlocked Pipeline Stages}
\newacronym{wan}{WAN}{Wide area network}
\newacronym{lan}{LAN}{Local area network}
\newacronym{man}{MAN}{Metropolitan are network}
\newacronym{usb}{USB}{Universal serial bus}
\newacronym[plural=OSes]{os}{OS}{Operating system}
\newacronym[plural=RTOSes]{rtos}{RTOS}{Real-time operating system}
\newacronym{ic}{IC}{integrated circuit}
\newacronym{eeprom}{EEPROM}{Electrically erasable programmable \acrshort{rom}}
\newacronym{pda}{PDA}{Personal digital assistant}
\newacronym{dsp}{DSP}{Digital signal processor}
\newacronym{spi}{SPI}{Serial peripheral interface}
\newacronym{asic}{ASIC}{Application-specific integrated circuit}
\newacronym{fpga}{FPGA}{Field-programmable gate array}
\newacronym{adc}{ADC}{Analog-to-digital converter}
\newacronym{dac}{DAC}{Digital-to-analog converter}
\newacronym{hw}{HW}{hardware}
\newacronym{sw}{SW}{software}
\newacronym{cpu}{CPU}{Central processing unit}
\newacronym{jtag}{JTAG}{Joint test action group}
\newacronym{sdr}{SDR}{Synchronous dynamic random access memory}
\newacronym{dram}{DRAM}{Dynamic random-access memory}
\newacronym{ddram}{DDR}{Double data rate synchronous \acrshort{dram}}
\newacronym{rf}{RF}{Radio frequency}
\newacronym{i2s}{I\textsuperscript{2}S}{Integrated Interchip Sound}
\newacronym{spdif}{S/PDIF}{Sony-Philips Digital Interface Format}
\newacronym{slic}{SLIC}{Subscriber line interface circuit}
\newacronym{ip}{IP}{Internet Protocol}
\newacronym{voip}{VOIP}{Voice over \acrshort{ip}}
\newacronym{pcm}{PCM}{Pulse code modulation}
\newacronym{lpcc}{LPCC}{Quad Flat No-leads}
\newacronym{lna}{LNA}{Low-noise amplifier}
\newacronym{pa}{PA}{Power amplifier}
\newacronym{gui}{GUI}{Graphical user interface}
\newacronym{cli}{CLI}{Command-line interface}
\newacronym{posix}{POSIX}{Portable operating system interface}
\newacronym{MB}{MB}{Mega-Byte}
\newacronym{bsd}{BSD}{Berkeley Software Distribution}
\newacronym{iec}{IEC}{International Electrotechnical Commission}
\newacronym{pcb}{PCB}{printed circuit board}
\newacronym{ddsn}{DDSN}{Dynamic Domain Name Service}
\newacronym{pwm}{PWM}{Pulse-width modulation}
\newacronym{ac}{AC}{Alternating current}
\newacronym{dc}{DC}{Direct current}
\newacronym{rms}{RMS}{Root-mean square}
\newacronym{tcp}{TCP}{Transmission Control Protocol}
\newacronym{tcpip}{TCP/IP}{\acrlong{tcp}/\acrlong{ip}}
\newacronym{thd}{THD}{Total Harmonic Distortion}
\newacronym{iot}{IoT}{Internet of Things}
\newacronym{sdio}{SDIO}{Secure Digital Input Output}
\newacronym{qfn}{QFN}{Quad Flat No-leads}
\newacronym{ssr}{SSR}{Solid-state relay}
\newacronym{hdd}{HDD}{Hard-disk drive}
\newacronym{ptc}{PTC}{Positive thermal coefficient}
\newacronym{rdbms}{RDBMS}{Relational Data-base management system}
\newacronym{tht}{THT}{Through-hole technology}
\newacronym{smt}{SMT}{Surface-mount technology}
\newacronym{i2c}{I\textsuperscript{2}C}{Inter-Integrated Circuit}
\newacronym{bom}{BOM}{Bill of the materials}
\newacronym{dns}{DNS}{Domain name server}
\newacronym{ddns}{DDNS}{Dynamic \acrlong{dns}}
\newacronym{nat}{NAT}{Network address translation}
\newacronym{smps}{SMPS}{Switch-mode power supply}

\newglossaryentry{ethernet}{
	name=ethernet,
	description={family of \gls{computer} networking technologies for \glspl{lan} and \glspl{man}, conforming to standard \gls{ieee} 802.3}
}
\newglossaryentry{firmware}{
	name=firmware,
	description={the combination of a \gls{hw} device, e.g. an \gls{ic}, and \gls{computer} instructions and data that reside as read only \gls{sw} on that device, it usually cannot be modified during normal operation of the device}
}
\newglossaryentry{flash}{
	name=flash,
	description={an electronic non-volatile \gls{computer} storage medium (memory) that can be electrically erased and reprogrammed, next evolution of \gls{eeprom}}
}
\newglossaryentry{linux}{
	name=linux,
	description={an \Gls{unix}-like and mostly \acrshort{posix}-compliant \gls{computer} \gls{os} assembled under the model of free and open-source \gls{sw} development and distribution, from the beginning maintained by Linus Torvalds},
	plural=linuces
}
\newglossaryentry{router}{
	name=router,
	description={a networking device that forwards data packets between \gls{computer} networks, connected to two or more data lines from different networks}
}
\newglossaryentry{system}{
	name=system,
	description={a set of interacting or interdependent components forming an integrated whole, observing properties not obtainable with individual components}
}
\newglossaryentry{kernel}{
	name=kernel,
	description={a \gls{computer} \gls{program} that manages \gls{io} requests from software, and translates them into data processing instructions for the central processing unit and other electronic components of a \gls{computer}, being a fundamental part of a modern \gls{computer}'s \gls{os}}
}
\newglossaryentry{shell}{
	name=shell,
	description={a user interface for access to an \gls{os}'s services, using either \gls{cli} or \gls{gui}, depending on a \gls{computer}'s role and particular operation}
}
\newglossaryentry{interface}{
	name=interface,
	description={a shared boundary across which two separate components of a \gls{computer} \gls{system} exchange information that can occur between \gls{sw}, \gls{computer} \gls{hw}, peripheral devices, humans and combinations of these}
}
\newglossaryentry{unix}{
	name=unix,
	description={a family of multi-taskings, multi-user \gls{computer} \gls{os} that derive from the original AT\&T Unix, developed in the 1970s at the Bell Labs research center by Ken Thompson, Dennis Ritchie, and others}
}
\newglossaryentry{android}{
	name=android,
	description={a mobile \gls{os} based on the \Gls{linux} \gls{kernel} and currently developed by Google, designed primarily for touchscreen mobile devices such as smartphones and tablet \glspl{computer}, and for specialized user \glspl{interface} like televisions (Android TV), cars (Android Auto), and wrist watches (Android Wear).}
}
\newglossaryentry{network}{
	name=network,
	description={a medium that allows computing devices pass data to each other along links (data connections)}
}
\newglossaryentry{utility}{
	name=utility,
	description={is \gls{system} \gls{sw} designed to help analyze, configure, optimize or maintain a \gls{computer}},
	plural=utilities
}
\newglossaryentry{library}{
	name=library,
	description={a collection of \glspl{program} and \gls{sw} packages made generally available, often loaded and stored on disk for immediate use}
}
\newglossaryentry{driver}{
	name=driver,
	description={a \gls{computer} \gls{program} that operates or controls a particular type of device that is attached to a \gls{computer}}
}
\newglossaryentry{compiler}{
	name=compiler,
	description={a \gls{computer} \gls{program} (or set of \glspl{program}) that transforms source code written in a programming language (the source language) into another \gls{computer} language (the target language, often having a binary form known as object code)}
}
\newglossaryentry{daemon}{
	name=daemon,
	description={a \gls{computer} \gls{program} running on the multi-tasking \glspl{os} in a background, rather than being under the direct control of an interactive user}
}
\newglossaryentry{command}{
	name=command,
	description={a directive to a \gls{computer} \gls{program} acting as an interpreter of some kind, in order to perform a specific task, commonly a directive to some kind of \gls{cli}, such as a \gls{shell}}
}
\newglossaryentry{computer}{
	name=computer,
	description={a programmable machine, that responds to a specific set of instructions in a well-defined manner and can execute a prerecorded list of instructions (a \gls{program}).}
}
\newglossaryentry{application}{
	name=application,
	description={a \gls{program}, or group of \glspl{program}, that is designed for the end user}
}
\newglossaryentry{program}{
	name=program,
	description={a specific set of ordered operations for a computer to perform}
}
\newglossaryentry{peripheral}{
	name=peripheral,
	description={a device that is connected to and works with a \gls{computer} in a some way, but is not essential to a \gls{computer}'s function}
}
\newglossaryentry{voltage}{
	name=voltage,
	description={also called electromotive force, is a quantitative expression of the potential difference in charge between two points in an electrical field}
}
\newglossaryentry{current}{
	name=current,
	description={(electric) is the flow of charged particles through a conducting medium}
}
\newglossaryentry{stack}{
	name=stack,
	description={(protocol) is an implementation of a computer networking protocol suite, used interchangeably}
}
\newglossaryentry{cloud}{
	name=cloud,
	description={(computing) is a model for enabling ubiquitous, convenient, on-demand access to a shared pool of configurable computing resources}
}
\newglossaryentry{datasheet}{
	name=datasheet,
	description={a document that summarizes the performance and other technical characteristics of a product, machine, component (e.g., an electronic component)}
}
\newglossaryentry{memory}{
	name=memory,
	description={In computing, refers to the computer hardware devices used to store information for immediate use}
}
\newglossaryentry{arduino}{
	name=arduino,
	description={common term for a software company, project, and user community that designs and manufactures computer open-source hardware, open-source software, and microcontroller-based kits for building digital devices and interactive objects that can sense and control physical devices}
}